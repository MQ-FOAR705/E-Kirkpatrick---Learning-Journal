\documentclass{article}
\usepackage[utf8]{inputenc}

\title{FOAR705 Learning Journal}
\author{Ellen Kirkpatrick }
\date{August 2019}

\begin{document}

\maketitle

\section{Week 1}
Met Brian and Shawn before the first class to set up the necessary platforms and data on my laptop, everything for the next few weeks is on the laptop successfully.
First half of lesson we talked as a class about our expectations of the class and our worries. I found the section of being digitally literate and computational thinking really interesting. I never had considered these concepts in this way. 
I am still a bit worried about this unit as I am not familiar with this area but am filled with more confidence after speaking to the conveners. They made me feel more at ease and seem very supportive and willing to help. 
Interested in the version control online repository as I tend to write multiple drafts and versions of a single assignment. This could be helpful. 

\section{Week 2}
Data carpentry workshop - formatting data and formatting problems 
What steps need to be taken to clean data:\\
•	Compile data into one single tab/table (do not change raw data - create a new tab) \\
•	Different Key Id’s for the Tanzania data so it is not read the same as the Mozambique data.\\
•	Make sure labels/headings are all consistent with one another (adding underscores).\\
•	Create columns for all the different variables - dwelling, livestock, water use, plots etc.\\
•	Enter the data consistently - same types of responses, single piece of info per square and same spelling.\\
•	For missing info leave blank and change -99 to blank also.\\
\\
\textbf{Cleaning data exercise:}\\
1.	Started new worksheet, kept Mozambique key id’s the same. Listed Tanzania key id’s 1-20 to avoid confusion. \\
2.	Put all variables into separate columns - dwelling, livestock, plots, water-use etc.\\
3.	Made spelling and titles consistent - adding underscores where necessary. Checked the data entries to make sure the spelling was consistent too.\\
4.	Started with Mozambique data first - re-entered the dwelling, livestock and other variables info. Had to add new columns for livestock (livestock unknown) and for dwelling (cowshed inc & barn inc). \\
5.	Problem:  Did not know how to deal with inconsistencies in livestock data. Transferred data as best I could but did not understand the columns with “looked after cow/poultry”. Added a 1 if there was a yes for these columns but they were not included in livestock numbers. \\
6.	Moved on to Tanzania data. Repeated same steps. \\
7.	Left blank where there was no data entered in the original data set - i.e. Tanzania plots and key id 10 livestock. \\
8.	Table re-formatted so all info on same spreadsheet. \\
9.	Saved as an additional sheet on raw data file, and saved as a separate file. Separate file upload to github.  \\
\\
https://github.com/MQ-FOAR705/E-Kirkpatrick---exercises\\
\\
{Error: Re-read formatting problems and realised I made mistakes with entering data. Where the animals listed did not match the total number owned from the raw data, I added a 1 to my newly created label of livestock unkown. This is wrong, as the total data reflects a mixture of the animals specified not an additional animal. Also, Mozambique data also had a number listed attached to ‘none’. Did not know how to interpret this so put it down as livestock unknown. I will need to re-do the livestock section for both countries to be more accurate.}\\
\\
\textbf{Reflection on exercise:}\\
Found it to be a very slow process where I was constantly going back and forth. I found it hard to know the best way to enter the data, so was referring to the online lessons. I feel there are a lot more errors which I am unable to identify. Sometimes got lost in where I was exactly entering the data and what was needed. Will need to practice this more in order to become more competent. Interested to talk through this with other people to see their results. \\

\textbf{Metadata exercise}\\
Discussed the solutions that are listed on data carpentry  as a group. 
Metadata that should be recorded:\\
- exact wording of questions used in interviews\\
- description of the data allowed in each column\\
-definitions of any categorical variables (especially those that are unknown). \\
- definitions of what was included or specified as a variable - e.g. what is a room, what is a plot etc. \\


\textbf{Problems with data in discipline (politics and IR):}\\
Data is often biased or skewed and only includes specific factors/variables. Often this may be to further political agendas or to provide statistics in support of a policy etc.
Producing only quantitative data may not show the complexity of the issues or human life, as well as the context.\\
\\
\textbf{Scoping assignment and Latex:}\\
Did not know how to approach this, so write down many responses to trigger questions until I could finally start narrowing down more specifically to my project. Started very broad and was missing the point. I think it is more specific now but I haven’t really thought about yet the technical side of my research area. \\
Not confident in using technical language or referring to the technical sides of the project as I feel I don’t know enough but tried my best.
The Latex software was easier to use than expected.\\
\textbf{Errors}\\
I tried a few times to code in order to perform functions - vertical line breaks, removing indents, changing text and style but failed. A classmate showed me later in the week how to create line breaks which I then implemented and how to make text bold. But still struggled with the others, even with the online guides\\
Struggled to find how to download as a .tex file. Had to ask for help. People on slack helped resolve. \\
Could not commit to Github. Kept being redirected to a payment page.
Downloaded as a pdf and re-uploaded to Github. Managed to resolve through changing online settings.\\
When doing the learning journal on overleaf - underscores in text made the formatting go crazy. Did not understand why. Removed underscores for the meantime. \\

\section{Week 3}\\
\textbf{Data carpentry exercises:}\\
Dates as data:\\
Aim: Separate dates into isolated components so they can are easier for the computer to understand. 
Created a new tab on the dates spreadsheet and copied over the interview dates column.\\
1. Created new columns for day, month and year.\\
2. Problem: I did not know how to perform the date functions outlined in exercise. Watched a youtube tutorial on applying it. Also learnt how to continue a formula down an entire column (clicking on the bottom right corner of the box and double clicking left mouse).\\
3. Applied formula for day, month and year.\\
4. Result: the date has spread over 3 separate columns, each with their own number value rather than be represented as a date. \\
5. Aim: Input new value to interview date column.\\
6. Inputted 17/11 as directed in interview date.\\
7. Problem: Automatically changed to ‘17-Nov’ format and the formula did not automatically populate other cells with day, month and year as specified in instructions. The year automatically went to 2019. Not sure whether to keep this row included in data  collection or not?\\
8. Solution: Always input the year, otherwise it will automatically choose the current year. Could not get solution for carrying across the formula, will have to ask someone. Double clicking the bottom right corner of a cell and then dragging down to include the columns of newly inputted information will carry across the function. \\
\\
Quality Assurance:\\
1. Created new spreadsheet in SAFI clean workbook.\\
2. Followed directions and inputted minimum and maximum value for whole number.\\
3. Entered a number outside of this range to test. Error message appeared. Successful.\\
4. Added an input message using data validation button from data toolbar to specify more instructions. \\
5. Played around with creating a warning message, but returned the setting to stop. \\
6. Aim: Apply new validation rule to another column in data.\\
7. Chose ‘rooms’ to apply new validation rule. \\
8. Applied a rule that data must be a whole number between 1 and 10.\\
9. Added an input message explaining what the column data is about and the parameters. Successful.\\
10. Changed error message to a warning message (to allow for the possibility that a household may have more than 10 rooms). \\
11. Inputted 12 at the bottom to test new settings. Warning message popped up asking if I would like to continue or not. Successful.\\
\\
Exporting data exercise:\\
1. Aim: Export spreadsheet and save as .csv file for more compatibility.\\
2. Tried to save spreadsheet as .csv file.\\
3. Error: Could not save the spreadsheet I was using for the clean data (with data validation rules). Not comptabile with csv.\\
4. Solution: Saved file with data validation as excel workbook. Saved a copy (which removed the rules) as .csv. Uploaded the data validation workbook (excel format) on github for proof of completion.\\
5. Saved the dates exercise as .csv. Had to make sure there was only one spreadsheet. No problems.\\
\\
\textbf{Scoping exercise II: computational analysis:}\\
Working with a group of 3 (Jeremy, John and myself)  on this project. We are in similar departments (philosophy and politics) and share  technical problems. We discussed these as a group and decided to work on this scoping exercise together as we shared the most important problem we could identify. \\
Aim: Finish scoping exercise II - add on to scoping exercise 1 and identify problems and take steps in thinking about how they can be solved. \\
1. Identified common problem - having to sort and read through multiple sources to identify common themes/arguments across them and to cross-check common references and citations used.  \\
2. Identified end goal - A singular program within which we can access multiple sources at once to identify relevant ideas or information and draw links between them. Cross-checks should be able to be performed to check for common citations. Sources should correspond to one another according to key terms and themes.\\
3.Discussed how to break down this problem into smaller steps for the decomposition: identify key themes — use key words for search — read abstracts - download sources — open sources — read to decide relevance — label — store — begin reading — cross-check references — identify main arguments — compile notes. \\
4. Discussed how to reach our end goal.\\
5. Broke down this process into steps for algorithm design: search database & find reports —> download relevant source — find key terms — save file — read sources — create hyperlinks — check common sources — compile relevant info.\\
6. Wrote down steps in an overleaf document.\\
7. Result: Committed to github. Saved as a pdf file onto hard drive and uploaded on cloudstor.\\ 
\\
\textbf{Reflection: scoping exercise II}\\
I struggled understanding what was necessary for this assignment. I tried multiple times but was missing the point of the algorithm design. I could break down the steps for the decomposition but not complete the second part.\\
I found working with fellow students who have similar problems really helped. As we discussed more and more what we find difficult in research, it slowly became more clear to me. I found working with other people we were all able to contribute something different and make a more clear and informed process.\\
\\
\textbf{Writing assignment on Latex:}\\
Aim: Write scoping exercise II on Latex and commit to Github.\\
Found using Latex a little bit easier this time as I had used it before. Could make line breaks and make text bold.\\
Learnt how to itemize text in order to create dot point lists which I did for this assignment.
The entire formatting/structure is still very simple as I have not delved too far into it. I also do not think it is necessary for this assignment.\\
Worry - I am worried about using Latex for longer pieces of work where additional coding and functions will be necessary as I am unfamiliar about using them and don’t know how to work on such big pieces of work. \\
This time no problems downloading assignments or committing to Github.

\end{document}
