\documentclass{article}
\usepackage[utf8]{inputenc}

\title{FOAR705 Learning Journal}
\author{Ellen Kirkpatrick }
\date{August 2019}

\begin{document}

\maketitle

\section{Week 1}
Met Brian and Shawn before the first class to set up the necessary platforms and data on my laptop, everything for the next few weeks is on the laptop successfully.
First half of lesson we talked as a class about our expectations of the class and our worries. I found the section of being digitally literate and computational thinking really interesting. I never had considered these concepts in this way. 
I am still a bit worried about this unit as I am not familiar with this area but am filled with more confidence after speaking to the conveners. They made me feel more at ease and seem very supportive and willing to help. 
Interested in the version control online repository as I tend to write multiple drafts and versions of a single assignment. This could be helpful. 

\section{Week 2}
Data carpentry workshop - formatting data and formatting problems 
What steps need to be taken to clean data:\\
•	Compile data into one single tab/table (do not change raw data - create a new tab) \\
•	Different Key Id’s for the Tanzania data so it is not read the same as the Mozambique data.\\
•	Make sure labels/headings are all consistent with one another (adding underscores).\\
•	Create columns for all the different variables - dwelling, livestock, water use, plots etc.\\
•	Enter the data consistently - same types of responses, single piece of info per square and same spelling.\\
•	For missing info leave blank and change -99 to blank also.\\
\\
\textbf{Cleaning data exercise:}\\
1.	Started new worksheet, kept Mozambique key id’s the same. Listed Tanzania key id’s 1-20 to avoid confusion. \\
2.	Put all variables into separate columns - dwelling, livestock, plots, water-use etc.\\
3.	Made spelling and titles consistent - adding underscores where necessary. Checked the data entries to make sure the spelling was consistent too.\\
4.	Started with Mozambique data first - re-entered the dwelling, livestock and other variables info. Had to add new columns for livestock (livestock unknown) and for dwelling (cowshed inc & barn inc). \\
5.	Problem:  Did not know how to deal with inconsistencies in livestock data. Transferred data as best I could but did not understand the columns with “looked after cow/poultry”. Added a 1 if there was a yes for these columns but they were not included in livestock numbers. \\
6.	Moved on to Tanzania data. Repeated same steps. \\
7.	Left blank where there was no data entered in the original data set - i.e. Tanzania plots and key id 10 livestock. \\
8.	Table re-formatted so all info on same spreadsheet. \\
9.	Saved as an additional sheet on raw data file, and saved as a separate file. Separate file upload to github.  \\
https://github.com/MQ-FOAR705/E-Kirkpatrick---exercises\\
{Error: Re-read formatting problems and realised I made mistakes with entering data. Where the animals listed did not match the total number owned from the raw data, I added a 1 to my newly created label of livestock unkown. This is wrong, as the total data reflects a mixture of the animals specified not an additional animal. Also, Mozambique data also had a number listed attached to ‘none’. Did not know how to interpret this so put it down as livestock unknown. I will need to re-do the livestock section for both countries to be more accurate.}\\
\\
\textbf{Reflection on exercise:}\\
Found it to be a very slow process where I was constantly going back and forth. I found it hard to know the best way to enter the data, so was referring to the online lessons. I feel there are a lot more errors which I am unable to identify. Sometimes got lost in where I was exactly entering the data and what was needed. Will need to practice this more in order to become more competent. Interested to talk through this with other people to see their results. \\
\\
\textbf{Problems with data in discipline (politics and IR):}\\
Data is often biased or skewed and only includes specific factors/variables. Often this may be to further political agendas or to provide statistics in support of a policy etc.
Producing only quantitative data may not show the complexity of the issues or human life, as well as the context.\\
\\
\textbf{Scoping assignment and Latex:}\\
Did not know how to approach this, so write down many responses to trigger questions until I could finally start narrowing down more specifically to my project. Started very broad and was missing the point. I think it is more specific now but I haven’t really thought about yet the technical side of my research area. \\
Not confident in using technical language or referring to the technical sides of the project as I feel I don’t know enough but tried my best.
The Latex software was easier to use than expected.
{Error- I tried a few times to code in order to perform functions (creating vertical line breaks and removing indents) but was unsuccessful}

\end{document}
